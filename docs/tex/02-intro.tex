\chapter*{Введение}
\addcontentsline{toc}{chapter}{ВВЕДЕНИЕ}
В современном мире авиаперевозки становятся все более популярным способом путешествия, что требует эффективной системы бронирования авиабилетов. Разработка системы бронирования авиарейсов имеет высокую актуальность в связи с растущим спросом на авиаперевозки и необходимостью обеспечения удобства и оперативности процесса бронирования.

Данный проект представляет собой разработку портала бронирования авиабилетов, функциональность которого включает в себя возможность поиска и выбора оптимальных авиарейсов по различным критериям (цена, время вылета), онлайн-бронирование билетов, управление бронированиями, а также использование системы лояльности. 

Цель работы -- hазработать прототип системы бронирования авиабилетов на базе веб-интерфейса.

Для достижения поставленной цели требуется решить следующие задачи:

\begin{itemize}
    \item сделать обзор существующих систем бронирования авиарейсов;
    \item описать назначение разработки;
    \item описать требования к разрабатываемой системе;
    \item разработать концептуальную модель системы в виде IDEF0;
    \item описать сценарии функционирования системы;
    \item привести диаграммы классов разрабатываемой системы;
    \item обосновать выбор СУБД, языка программирования и используемых библиотек;
    \item разработать ПО бронирования авиарейсов;
    \item привести примеры работы ПО.
\end{itemize}
