\chapter{Технологический раздел}

\section{Выбор системы развертывания компонентов распределенной системы}


Согласно требованиям технического задания, разрабатываемая система должна быть распределенной. Характерной особенностью распределенных систем является высокое многообразие используемых технологий. Особенно непростая ситуация возникает, когда разные компоненты системы используют разные версии одной и той же библиотеки. Для того чтобы компоненты не конфликтовали друг с другом, необходимо ввести требование изолированности. В этом случае приходится использовать отдельные серверы, что может быть экономически нецелесообразно, либо использовать контейнеризацию. 

Контейнеризация -- это методология разработки и управления приложениями, которая позволяет упаковывать приложение и все его зависимости в изолированный контейнер (образ изолируемой части системы, содержащий приложение со всеми его зависимостям). Контейнеры обеспечивают среду выполнения для приложения, которая полностью отделена от других контейнеров и основной системы. Это облегчает развертывание, масштабирование и управление приложениями, а также обеспечивает надежность и безопасность при работе в различных средах. 

В качестве платформы для автоматизации развертывания, масштабирования и управления контейнеризированными приложениями на серверах было решено использовать Kubernetes. Такое решение было принято в результате следующего сравнительного анализа.

Существует несколько аналогов Kubernetes, которые также предоставляют возможности для управления контейнеризированными приложениями:
\begin{enumerate}
    \item Docker Swarm -- это оркестратор контейнеров, разработанный Docker, который позволяет управлять кластером Docker-хостов и запускать контейнеры в них. Он более прост в использовании по сравнению с Kubernetes, но может быть менее мощным в некоторых аспектах.
    \item Apache Mesos -- это распределенная система управления ресурсами, которая также поддерживает запуск контейнеров. Он предоставляет более общий подход к управлению ресурсами и приложениями, чем Kubernetes.
    \item Amazon ECS (Elastic Container Service) -- это управляемый сервис от Amazon Web Services для запуска и управления контейнерами на инфраструктуре AWS. Он предлагает простой способ запуска и масштабирования контейнеров без необходимости управления инфраструктурой.
\end{enumerate}

Достоинства Kubernetes по сравнению с аналогами:
\begin{itemize}
    \item Мощные возможности оркестрации: Kubernetes предоставляет широкий спектр возможностей для управления и автоматизации развертывания приложений в контейнерах.
    \item Большое сообщество и экосистема: Kubernetes имеет активное сообщество разработчиков и широкий выбор инструментов и плагинов для расширения его функциональности.
    \item Поддержка различных облачных и локальных сред: Kubernetes поддерживает различные облачные провайдеры и может быть развернут как локально, так и в облаке.
\end{itemize}

\section{Выбор операционной системы}
Согласно требованиям технического задания, разрабатываемый портал должен обладать высокой доступностью, работать на типичных архитектурах ЭВМ (Intel x86, Intel x64), а так же быть экономически недорогим для сопровождения. Таким образом, можно сформулировать следующие требования к операционной системе:
\begin{itemize}
    \item Распространенность. На рынке труда должно быть много специалистов, способных администрировать распределенную систему, работающую под управлением выбранной операционной системы.
    \item Надежность. Операционная система должна широко использоваться в стабильных проектах, таких как Mail.Ru, Vk.com, Google.com. Эти компании обеспечивают высокую работоспособность своих сервисов, и на их опыт можно положиться.
    \item Наличие требуемого программного обеспечения. Выбор операционной системы недолжен ограничивать разработчиков в выборе программного обеспечения, библиотек.
    \item Цена.
\end{itemize}

Под данные требования лучше всего подходит ОС Ubuntu. Ubuntu -- это дистрибутив, использующий ядро Linux. Как и все дистрибутивы Linux, Ubuntu является ОС с открытым исходным кодом, бесплатным для использования. 

\section{Выбор СУБД}

В качестве СУБД была выбрана PostgreSQL \cite{bib:postgres}, так как она наилучшим образом подходит под требования разрабатываемой системы:
\begin{itemize}
    \item Масштабируемость: PostgreSQL поддерживает горизонтальное масштабирование, что позволяет распределить данные и запросы между несколькими узлами базы данных. Это особенно полезно в географически распределенных системах, где данные и пользователи могут быть разбросаны по разным регионам.
    \item Географическая репликация: PostgreSQL предоставляет возможность настройки репликации данных между различными узлами базы данных, расположенными в разных географических зонах. Это позволяет обеспечить отказоустойчивость и более быстрый доступ к данным для пользователей из разных частей мира.
    \item Гибкость и функциональность: PostgreSQL обладает широким набором функций и возможностей, что делает его подходящим для различных типов приложений и использования в распределенной среде. Он поддерживает сложные запросы, транзакции, хранимые процедуры и многое другое.
    \item Надежность и отказоустойчивость: PostgreSQL известен своей надежностью и стабильностью работы. В распределенной географической системе это особенно важно, поскольку он способен обеспечить сохранность данных и доступность даже при сбоях в отдельных узлах.
\end{itemize}

\section{Выбор языка разработки и фреймворков компонент портала}
Проанализируем техническое задание на разработку портала. Исходя из приведенных требований к системе, можно выявить требования к языку программирования:
\begin{itemize}
    \item Совместимость с выбранными ранее технологиями. Выбранный язык должен уметь взаимодействовать с ОС Linux, СУБД PostgreSQL.
    \item Производительность: C++ является компилируемым языком программирования, что позволяет создавать быстродействующие приложения. Он обладает низким уровнем абстракции, что позволяет разработчику более тонко управлять ресурсами и оптимизировать производительность программы.
    \item Расширяемость: C++ обладает возможностью использовать объектно-ориентированный подход к программированию, что делает его удобным для создания сложных и масштабируемых систем.
\end{itemize}

Выбор oatpp \cite{bib:oat} в качестве фреймворка для разработки программного обеспечения также имеет свои преимущества и может быть обоснован следующими аспектами:
\begin{itemize}
    \item Высокая производительность: oatpp является легковесным и быстрым фреймворком, который спроектирован для обеспечения высокой производительности приложений. Он оптимизирован для работы с сетевыми запросами и обработки данных, что позволяет создавать эффективные веб-сервисы.
    \item Поддержка многопоточности: oatpp предоставляет удобные инструменты для работы с многопоточностью, что позволяет создавать параллельные и распределенные приложения. Это особенно важно для систем, где требуется обработка большого количества запросов одновременно.
    \item RESTful API: oatpp поддерживает разработку RESTful API, что делает его удобным выбором для создания веб-сервисов и API. Он предоставляет инструменты для удобной маршрутизации запросов, валидации данных и других задач, связанных с разработкой веб-приложений.
    \item Модульность и расширяемость: oatpp построен на основе модульной архитектуры, что позволяет легко добавлять новый функционал и расширять возможности фреймворка. Это делает его гибким инструментом для разработки различных типов приложений.
\end{itemize} 

Для разработки интерфейсной части портала был выбран стек: React \cite{bib:react} и typescript \cite{bib:typescript}.

\section{Обеспечение надежности портала}


В данной системе для обеспечения надежности функционирования СУБД будет применяться репликация и шардинг. Для обеспечения надежности данных СУБД необходимо разработать скрипт для автоматического создания резервной копий базы данных по расписанию.

Для фронтенда и бекендов целесообразно применить зеркалирование. Это обеспечит отказоустойчивость системы: в случае сбоя любого из ее узлов запросы на чтение данных будут выполняться. 

\section{Примеры работы ПО}

Ниже приведены основные формы портала. 
\newpage

\begin{figure}[h!]
  \centering
  \includegraphics[width = \linewidth]{inc/img/login.png}
  \caption{Форма входа}
  \label{fig:scheme}
\end{figure}

\begin{figure}[h!]
  \centering
  \includegraphics[width = \linewidth]{inc/img/registration.png}
  \caption{Форма регистрации пользователя}
  \label{fig:scheme}
\end{figure}

\begin{figure}[h!]
  \centering
  \includegraphics[width = \linewidth]{inc/img/flights.png}
  \caption{Форма с выбором авиабилетов}
  \label{fig:scheme}
\end{figure}


\begin{figure}[h!]
  \centering
  \includegraphics[width = \linewidth]{inc/img/flight.png}
  \caption{Форма с покупкой авиабилетов}
  \label{fig:scheme}
\end{figure}



\begin{figure}[h!]
  \centering
  \includegraphics[width = \linewidth]{inc/img/lk.png}
  \caption{Форма с аккаунтом пользователя и бронированиями}
  \label{fig:scheme}
\end{figure}


