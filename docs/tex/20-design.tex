\chapter{Конструкторский раздел}


\section{Концептуальная модель системы в нотации IDEF0}


Для создания функциональной модели портала, отражающей его основные функции и потоки информации наиболее наглядно использовать нотацию IDEF0. На рисунке \ref{img:idef0-0} приведена концептуальная модель системы. На рисунке \ref{img:idef0-1} представлена декомпозиция функциональной модели системы.

 \begin{figure}[h!]
  \centering
  \includegraphics[width = \linewidth]{inc/img/idef0-0.pdf}
  \caption{Концептуальная модель в нотации IDEF0}
  \label{img:idef0-0}
\end{figure}

 \begin{figure}[h!]
  \centering
  \includegraphics[width = \linewidth]{inc/img/idef0-1.pdf}
  \caption{Декомпозиция функциональной модели системы}
  \label{img:idef0-1}
\end{figure}


\section{Сценарии функционирования системы}


Для детальной разработки портала используется унифицированный язык моделирования UML. В системе выделены 2 роли: администратор и пользователь (авторизованный и неавторизованный). На рисунках \ref{img:use-case-admin}-\ref{img:use-case-user} представлены диаграммы прецедентов для выделенных ролей и описаны сценарии функционирования наиболее значимых прецедентов.


 \begin{figure}[h!]
  \centering
  \includegraphics[width = \linewidth]{inc/img/use-case-user.pdf}
  \caption{Диаграмма прецедентов для роли <<пользователь>>}
  \label{img:use-case-user}
\end{figure}

 \begin{figure}[h!]
  \centering
  \includegraphics[width = \textwidth / 3]{inc/img/use-case-admin.pdf}
  \caption{Диаграмма прецедентов для роли <<администратор>>}
  \label{img:use-case-admin}
\end{figure}


\subsubsection*{Регистрация пользователя}
\begin{enumerate}
    \item Пользователь переходит на страницу входа с помощью кнопки <<войти>>, либо автоматически перенаправляется на соответствующую страницу при попытке совершения действий, которые невозможно совершить без регистрации (например, покупка или возврат билетов). 
    \item Пользователь нажимает на кнопку <<зарегистрироваться>> и перенаправляется на страницу регистрации, где вводит различные наборы полей. Валидация входных данных осуществляется <<на лету>> на стороне пользователя. При отправке данных на фронтенд, он тоже производит валидацию.
    \item Пользователь нажимает кнопку «Регистрация» и перенаправляется на главную страницу портала.
\end{enumerate}

\subsubsection*{Авторизация на портале}
\begin{enumerate}
    \item Пользователь переходит на страницу входа с помощью кнопки <<войти>>, либо автоматически перенаправляется на соответствующую страницу при попытке совершения действий, которые невозможно совершить без регистрации (например, покупка или возврат билетов). 
    \item Вводит учётные данные, нажимает кнопку <<войти>>.
    \item Пользователь даёт согласие на использование его данных. Если пользователь не дает согласия, то он перенаправляется на страницу с ошибкой.
    \item Пользователь перенаправляется на главную страницу портала.
\end{enumerate} 

\subsubsection*{Просмотр доступных для покупки билетов}
Сценарий доступен как для авторизованного, так и для неавторизованного пользователя.
\begin{enumerate}
    \item Пользователь задаёт параметры поиска авиабилета (город отправления, город назначения, дату поездки) и нажимает кнопку <<найти>>.
    \item На экране появляется список доступных билетов с детальной информацией о билете (цена, время вылета).
\end{enumerate}

\subsubsection*{Покупка билета}
Сценарий доступен только для авторизованного пользователя.
\begin{enumerate}
    \item Пользователь задаёт параметры поиска авиабилета (город отправления, город назначения, дату поездки) и нажимает кнопку <<найти>>.
    \item На экране появляется список доступных билетов с детальной информацией о билете (цена, время вылета).
    \item Пользователь выбирает билет, переходит на страницу с детальной информацией о билете и нажимает кнопку <<купить>>.
    \item Пользователю предлагается выбрать: начислить баллы за покупаемый билет или списать баллы с бонусного счёта для уменьшения цены билета.
    \item С бонусным счётом производится операция в соответствии с выбранном на предыдущем шаге действием пользователя.
\end{enumerate}

\subsubsection*{Возврат билета}
Сценарий доступен только для авторизованного пользователя.
\begin{enumerate}
    \item Пользователь выбирает кнопку <<личный кабинет>> на главной странице портала.
    \item В разделе <<купленные билеты>> пользователь выбирает билет, который хочет вернуть, и нажимает соответствующую кнопку.
    \item С бонусным счётом производятся дейтсвия, обратные к совершённым при покупке билета. При этом остаток на бонусном счёт не может быть отрицательным.
\end{enumerate}

\subsubsection*{Просмотр операций с бонусным счётом}
Сценарий доступен только для авторизованного пользователя.
\begin{enumerate}
    \item Пользователь выбирает кнопку <<личный кабинет>> на главной странице портала.
    \item Пользователь открывает раздел <<история операций с бонусным счётом>> и перенаправляется на страницу с историей.
\end{enumerate}

\subsubsection*{Получение статистики}
\begin{enumerate}
    \item Пользователь с ролью <<администратор>> нажимает на кнопку <<посмотреть историю запросов>> и перенаправляется на соответствующую страницу.
	\item Пользователь с ролью <<администратор>> нажимает на кнопку <<Получить статистику>>.
	\item Пользователь перенаправляется на страницу просмотра статистики о запросах.
\end{enumerate}

\subsubsection*{Спецификация сценария покупки билета}
Нормальный ход сценария.
\begin{table}[!h]
	\begin{center}
		\caption{\label{spec_buy_ticket}Спецификация покупки билета} 
		\footnotesize
		\begin{tabular}{|l|l|}
			\hline	
   \multicolumn{1}{|c|}{\begin{tabular}[c]{@{}c@{}}Действия актера\end{tabular}} & 
    \multicolumn{1}{c|}{\begin{tabular}[c]{@{}c@{}} Отклик системы\end{tabular}}  \\
\hline выбор билета из списка доступных & открытие страницы с \\
& подробной информацией о билете \\
\hline запрос покупки билета & успешная проверка авторизации и запрос подтверждения операции \\
\hline подтверждение операции & осуществление покупки \\
\hline
	\end{tabular}
	\end{center}
\end{table}

Альтернативный ход сценария.
\begin{table}[!h]
	\begin{center}
		\caption{\label{spec_buy_ticket}Спецификация покупки билета} 
		\footnotesize
		\begin{tabular}{|l|l|}
			\hline	
   \multicolumn{1}{|c|}{\begin{tabular}[c]{@{}c@{}}Действия актера\end{tabular}} & 
    \multicolumn{1}{c|}{\begin{tabular}[c]{@{}c@{}} Отклик системы\end{tabular}}  \\
\hline выбор билета из списка доступных & открытие страницы с \\
& подробной информацией о билете \\
\hline запрос покупки билета & успешная проверка авторизации и запрос подтверждения операции \\
\hline отклонение операции & покупка не осуществляется \\
\hline
	\end{tabular}
	\end{center}
\end{table}
\newpage
Альтернативный ход сценария.
\begin{table}[!h]
	\begin{center}
		\caption{\label{spec_buy_ticket}Спецификация покупки билета} 
		\footnotesize
		\begin{tabular}{|l|l|}
			\hline	
   \multicolumn{1}{|c|}{\begin{tabular}[c]{@{}c@{}}Действия актера\end{tabular}} & 
    \multicolumn{1}{c|}{\begin{tabular}[c]{@{}c@{}} Отклик системы\end{tabular}}  \\
\hline выбор билета из списка доступных & открытие страницы с \\
& подробной информацией о билете \\
\hline запрос покупки билета & неуспешная проверка авторизации,  \\
& перенаправление на страницу ввода учётных данных \\
\hline
 ввод учётных данных  & успешная проверка учётных данных,  \\
 & запрос подтверждения операции \\
\hline подтверждение операции & осуществление покупки\\
\hline
	\end{tabular}
	\end{center}
\end{table}


\section{Диаграммы классов}


Иерархии классов для разработки серверных приложений представлены в виде диаграммы классов:
\begin{itemize}
    \item сервиса рейсов --  на рисунке \ref{img:flights_uml};
    \item сервиса билетов --  на рисунке \ref{img:ticket_uml};
    \item сервиса сессий и пользователей --  на рисунке \ref{img:person_uml};
    \item сервиса программы лояльности --  на рисунке \ref{img:bonus_uml};
    \item сервиса-координатора --  на рисунке \ref{img:gateway_uml};
    \item сервиса статистики --  на рисунке \ref{img:stat_uml}.
\end{itemize}



\begin{figure}[h!]
  \centering
  \includegraphics[width = \linewidth]{inc/img/flights_uml.png}
  \caption{Диаграмма классов сервиса рейсов}
  \label{img:flights_uml}
\end{figure}

\begin{figure}[h!]
  \centering
  \includegraphics[width = \linewidth]{inc/img/ticket_uml.png}
  \caption{Диаграмма классов сервиса билетов}
  \label{img:ticket_uml}
\end{figure}


\begin{figure}[h!]
  \centering
  \includegraphics[width = \linewidth]{inc/img/person_uml.png}
  \caption{Диаграмма классов сервиса сессий и пользователей}
  \label{img:person_uml}
\end{figure}

\begin{figure}[h!]
  \centering
  \includegraphics[width = \linewidth]{inc/img/bonus_uml.png}
  \caption{Диаграмма классов сервиса программы лояльности}
  \label{img:bonus_uml}
\end{figure}

\begin{figure}[h!]
  \centering
  \includegraphics[width = \linewidth / 2]{inc/img/gateway_uml.png}
  \caption{Диаграмма классов сервиса-координатора}
  \label{img:gateway_uml}
\end{figure}

\begin{figure}[h!]
  \centering
  \includegraphics[width = \linewidth / 2]{inc/img/stat_uml.png}
  \caption{Диаграмма классов сервиса статистики}
  \label{img:stat_uml}
\end{figure}

\newpage
\subsubsection{Описание классов сервисов}

Сервисы рейсов, билетов, программы лояльности и сессий и пользователей спроектированы похожим образом. Они имеют:
\begin{itemize}
    \item слой доступа к данным, реализованный с помощью паттерна <<Репозиторий>> для абстракции хранения, а также паттерна <<пул объектов>> для эффективного использования активных подключений к базе данных;
    \item слой логики работы сервиса;
    \item слой интерфейса -- слой связи с другими сервисами с помощью http-запросов.
\end{itemize}

Так как сервисы спроектированы одинаково, часть классов совпадают. Опишем их:
\begin{itemize}
    \item IDAFacade -- абстрактный интерфейс слоя доступа к данным; PGDAFacade -- реализация этого абстрактного интерфейса для работы с данными, хранящимися под управлением СУБД Postgres;
    \item IDAFactory -- абстрактный интерфейс фабрики объектов, относящихся к библиотеке работы с данными (интерфейс спроектирован в соответствии с паттерном <<фабрика объектов>>); PGDAFactory -- реализация этого абстрактного интерфейса для работы с данными, хранящимися под управлением СУБД Postgres;
    \item PGConnection -- пул активных подлключений к базе данных, спроектированный в соответствии с паттерном <<пул объектов>>;
    \item IBLFacade -- абстрактный интерфейс слоя логики; BLFacade -- реализация этого абстрактного интерфейса;
    \item IServer -- абстрактный интерфейс серверного приложения, предоставляющего интерфейс, описанный HTTPController; Server -- реализация этого интерфейса.
    \item HTTPController -- класс, описывающий набор  HTTP-методов, которые доступны на сервисе. Фактически занимается распаковкой данных, пришедших по сети, заполнением необходимых структур этими данными и вызовом функций, реализующих логику работы методов. Также занимается подготовкой результирующих данных к отправке по сети после выполнения запроса. 
\end{itemize}

Кроме того, каждый сервис имеет собственные классы. Опишем их.

\subsubsection{Описание классов сервиса рейсов}

\begin{itemize}
    \item Flight -- класс, описывающий рейс, имеет поля:
    \begin{itemize}
        \item дата и время рейса;
        \item цена рейса;
        \item номер рейса;
        \item аэропорты вылета и прилёта;
        \item города вылета и прилёта.
    \end{itemize}
    \item IFlightRepository -- абстрактный интерфейс для работы с данными рейсов (интерфейс спроектирован в соответствии с паттерном <<репозиторий>>). PGFlightRepository -- реализация этого абстрактного интерфейса для работы с данными, хранящимися под управлением СУБД Postgres.
\end{itemize}

\subsubsection{Описание классов сервиса билетов}

\begin{itemize}
    \item Ticket -- класс, описывающий билет, имеет поля:
    \begin{itemize}
        \item номер билета;
        \item цена билета;
        \item номер рейса;
        \item статус (оплачен или отменён);
        \item идентификатор билета;
        \item владелец билета.
    \end{itemize}
    \item ITicketRepository -- абстрактный интерфейс для работы с данными билетов (интерфейс спроектирован в соответствии с паттерном <<репозиторий>>). PGTicketRepository -- реализация этого абстрактного интерфейса для работы с данными, хранящимися под управлением СУБД Postgres.
\end{itemize}

\subsubsection{Описание классов сервиса сессий и пользователей}

\begin{itemize}
    \item User -- класс, описывающий пользователя, имеет поля:
    \begin{itemize}
        \item идентификатор пользователя;
        \item логин пользователя;
        \item роль пользователя;
        \item возраст пользователя;
        \item пароль пользователя (в виде хэша).
    \end{itemize}
    \item Session -- класс, описывающий сессию пользователя, является JWT-token-ом;
    \item Credentials -- класс, который хранит логин и пароль пользователя.
    \item IUserRepository -- абстрактный интерфейс для работы с данными пользователей и сессий (интерфейс спроектирован в соответствии с паттерном <<репозиторий>>). PGUserRepository -- реализация этого абстрактного интерфейса для работы с данными, хранящимися под управлением СУБД Postgres.
\end{itemize}

\subsubsection{Описание классов сервиса программы лояльности}

\begin{itemize}
    \item BuyRequest -- класс, описывающий запрос на покупку билета, имеет поля:
    \begin{itemize}
        \item решение пользователя о списании или о начислении бонусов;
        \item цена билета;
        \item номер билета;
        \item имя пользователя.
    \end{itemize}
    \item BuyResponse -- класс, описывающий ответ на покупку билета, имеет поля:
    \begin{itemize}
        \item баланс счёта программы лояльности после покупки билета;
        \item количество денег, списанное за счёт бонусов;
        \item количество денег, списанное за счёт средств пользователя;
        \item статус счёта лояльности (золотой, серебряный, бронзовый) после покупки.
    \end{itemize}
    \item BalanceResponse -- класс, описывающий ответ на запрос информации о бонусном счёта и истории операций бонусного счёта, имеет поля:
    \begin{itemize}
        \item баланс счёта программы лояльности;
        \item статус счёта лояльности (золотой, серебряный, бронзовый);
        \item история операций (каждая запись содержит: абсолютное значение изменения баланса, дату операции, типа операции -- списание или начисление, идентификатор билета).
    \end{itemize}
    \item IBonusRepository -- абстрактный интерфейс для работы с данными программы лояльности (интерфейс спроектирован в соответствии с паттерном <<репозиторий>>). PGBonusRepository -- реализация этого абстрактного интерфейса для работы с данными, хранящимися под управлением СУБД Postgres.
\end{itemize}

\subsubsection{Описание классов сервиса-координатора}

\begin{itemize}
    \item HTTPController -- класс, описывающий набор  HTTP-методов, которые доступны на сервисе. Фактически предасвляет собой весь программный интерфейс системы. В своей работе для обслуживания приходящих запросов сервис использует интерфейсы сервисов программы лояльности (BonusService), рейсов (FlightService) и билетов (TicketService). Кроме того, сервис перенаправляет статистику запросов в очередь Kafka с помощью интерфейса KafkaStatistcsService.
    \item Server -- реализация серверного приложения, предоставляющего интерфейс, описанный HTTPController.
\end{itemize}

\subsubsection{Описание классов сервиса оплаты}

\begin{itemize}
    \item HTTPController -- класс, описывающий набор  HTTP-методов, которые доступны на сервисе (покупка и возврат билетов). В своей работе для обслуживания приходящих запросов сервис использует сервисы банка для выполнения транзакций со счётом пользователя.
    \item Server -- реализация серверного приложения, предоставляющего интерфейс, описанный HTTPController.
\end{itemize}

\subsubsection{Описание классов сервиса статистики}
\begin{itemize}
    \item HTTPController -- класс, описывающий набор  HTTP-методов, которые доступны на сервисе (просмотр истории запросов к системе и метрики, рассчитанные на основе этой истории). В своей работе для обслуживания приходящих запросов сервис использует очередь Kafka.
    \item Server -- реализация серверного приложения, предоставляющего интерфейс, описанный HTTPController.
\end{itemize}

\section{Диаграмма деятельности}

На рисунке \ref{img:dpd} изображена диаграмма деятельности при покупке билета. 

\begin{figure}[h!]
  \centering
  \includegraphics[width = 0.65\linewidth]{inc/img/dpd.pdf}
  \caption{Диаграмма деятельности при покупке билета}
  \label{img:dpd}
\end{figure}

\newpage
\section{Высокоуровневый дизайн пользовательского интерфейса}

Пользовательский интерфейс в разрабатываемой системе представляет собой Web-интерфейс, доступ к которому осуществляется через браузер (тонкий клиент).

Страница портала состоит из <<шапки>> (верхней части страницы, в которой находится логотип и верхнее меню со ссылками на основные разделы портала), основной части и <<футера>> (нижней части страницы, в которой обычно размещают ссылки на редко посещаемые, но необходимые, страницы, например, страницы с пользовательским соглашением).

Обобщенно структуру страниц портала можно представить следующим образом:
\begin{itemize}
    \item страница с обучением для пользователя;
    \item главная страница с поиском авиабилетов;
    \item страница подробной информацией о билете и кнопкой покупки;
    \item личный кабинет пользователя (информация о пользователе с возможностью ее изменить);
    \item страница с купленными билетами (с возможностью вернуть еще не использованные билеты);
    \item страница программы лояльности (история операций бонусного счета, остаток на бонусном счёте);
    \item страница входа;
    \item страница регистрации;
    \item страница со статистикой запросов в приложении (доступно администраторам.
\end{itemize}

