\chapter{Исследовательский раздел}
В данном разделе проводится исследование значения функционала качества на разных моделях. Все исследования проводились на данных об успеваемости студентов. 

\section{Исследование функционала качества  разных моделей}

Для исследования были выбраны следующие модели:
\begin{itemize}
    \item линейная модель;
    \item случайный лес с параметрами (100 деревьев, без обрезки);
    \item метод k ближайших соседей (k = 5);
    \item модель градиентного бустинга с параметрами, выбранными в предыдущем разделе
    (learning\_rate: 0.01; n\_estimators: 500; subsample: 0.80; max\_depth: 9).
    \item многослойный персептрон с 2 слоями: 29-64-20 нейронов.
\end{itemize}

Результаты можно видеть в таблице \ref{tbl_res}.

\begin{table}[h!]
	\begin{center}
		\caption{\label{tbl_res} Результаты исследоавания} 
		\footnotesize
		\begin{tabular}{|l|l|l|l|l|}
			\hline	
   \multicolumn{1}{|c|}{\begin{tabular}[c]{@{}c@{}} Модель \end{tabular}} & 
    \multicolumn{1}{c|}{\begin{tabular}[c]{@{}c@{}}MAPE\end{tabular}} \\
\hline линейная модель & 22.457 \\
\hline k-соседей & 26.374 \\
\hline случайный лес & 21.381 \\
\hline многослойный персептрон & 22.748 \\
\hline градиентный бустинг & 20.958 \\
\hline
	\end{tabular}
	\end{center}
\end{table}

Видно, что модель градиентного бустинга работает точнее остальных с точки зрения функционала качества. 
\section*{Вывод}
\addcontentsline{toc}{section}{Вывод}
В данном разделе было проведено исследование значения функционала качества на разных моделях. 
В результате исследования было выяснено, что модель градиентного бустинга работает точнее остальных с точки зрения функционала качества. 